%!TEX TS-program = xelatex
\documentclass[letterpaper, 11pt]{article}
    %--------DEPENDENCIES--------
        \usepackage{fontspec}
        \usepackage{latexsym}
        \usepackage[empty]{fullpage}
        \usepackage{titlesec}
        \usepackage{marvosym}
        \usepackage[usenames,dvipsnames,svgnames,table]{xcolor}
        \usepackage{verbatim}
        \usepackage{enumitem}
        \usepackage{hyperref}
        \usepackage{tabulary}
        \usepackage{tabularx}
        \usepackage{xspace}
        \usepackage{enumitem}
    %----------------------------

    \hypersetup{colorlinks=false,}

    %----SECTION-FORMATTING-----
        \titleformat{\section}{
        \vspace{-10pt}\raggedright\large
        }{}{0em}{}[\color{black}\titlerule \vspace{-5pt}]


    %---------------------------

    %--------FONT-SETUP---------
        \setmainfont[
            Path = fonts/,
            Extension =.ttf,
            UprightFont = *-Regular,
            ItalicFont = *-Italic,
            BoldFont = *-Bold,
            BoldItalicFont = *-BoldItalic,
        ]{Roboto}
        \newfontfamily\robotoLight[
            Path = fonts/,
            Extension =.ttf,
            UprightFont = *-Regular,
            ItalicFont = *-Italic,
        ]{RobotoLight}
        \newfontfamily\robotoMed[
            Path = fonts/,
            Extension =.ttf,
            UprightFont = *-Regular,
            ItalicFont = *-Italic,
        ]{RobotoMedium}
        \newfontfamily\robotoThin[
            Path = fonts/,
            Extension =.ttf,
            UprightFont = *-Regular,
            ItalicFont = *-Italic,
        ]{RobotoThin}
    %---------------------------

    %----------MARGINS----------
        \addtolength{\oddsidemargin}{-0.5in}
        \addtolength{\evensidemargin}{-0.375in}
        \addtolength{\textwidth}{1in}
        \addtolength{\topmargin}{-.75in}
        \addtolength{\textheight}{1.0in}
    %---------------------------

    %-----HEADER-SETUP---------
        %\pagestyle{fancy}
        %\fancyhf{} % clear all header and footer fields
        %\fancyfoot{}
        %\renewcommand{\headrulewidth}{0pt}
        %\renewcommand{\footrulewidth}{0pt}
    %---------------------------

    %--------NEW-COMMANDS-------
        %Creates tab command that tabs the text over the argument amount
            \newcommand{\tab}{\hspace*{.3cm}}
        %Accents Subsection Headings and List Item Titles
            \renewcommand{\accent}[1]{\textbf{\textcolor{MidnightBlue}{#1}}}
        %Resume Subsection-Takes 5 Arguments [optional:link]{Title}{Location}{Description}{Duration}
            \newcommand{\resumeSubsection}[5][]{
                \item[]
                    \newcolumntype{R}{>{\raggedleft\arraybackslash}X}
                    \begin{tabularx}{1\textwidth}{p{5.5in}R}
                        \accent{#2}\hspace{.25cm}\footnotesize\robotoLight{\href{https://#1}{#1}} & {\mbox{{#3}}} \\
                        \small\robotoLight{#4} & \robotoLight{\mbox{#5}}
                    \end{tabularx}
            }
        %Project Items-Takes 4 arguments {Title}{Language}{github}{Description}
            \newcommand{\project}[5]{
                    \item[]
                        \newcolumntype{R}{>{\raggedleft\arraybackslash}X}
                        \begin{tabularx}{1\textwidth}{p{2.1in}p{3.2in}R}
                            \accent{\mbox{#1}} & &  \mbox{#2}\\
                            \footnotesize\robotoThin{\href{https://#3}{#3}} & \small\robotoLight{#4} & \mbox{\robotoThin{#5}}\\
                        \end{tabularx}
            }
        %Resume Listed Item-Takes 2 arguments {Title}{Description}
            \newcommand{\resumeItem}[2]{\item[]\tab{\accent{#1}: \robotoLight{#2}}}
        %Commands for List Environment
            \newcommand{\resumeSubHeadingListStart}{\begin{itemize}[leftmargin=*]}
            \newcommand{\resumeSubHeadingListEnd}{\end{itemize}}


    %---------------------------

    \setlist{nosep}
    \begin{document}
        %------------HEADER---------

            \hspace{-0.65cm}\huge{\accent{Seth Gower}}
            \normalsize\newline
            \robotoLight{\href{mailto:sdg7234@rit.edu}{sdg7234@rit.edu}}\newline
            % \robotoLight{\href{https://sethgower.com}{sethgower.com}}\newline
            (757)553-0576\newline
            \robotoLight{\href{https://github.com/sethgower}{github.com/SethGower}}\newline
            \robotoMed Seeking Software/Hardware Engineering Coop, with
            Embedded systems for Spring-Summer 2021
        %---------------------------

        %--------EDUCATION----------
            \section{Education}
                \resumeSubHeadingListStart
                    \resumeSubsection
                        {Rochester Institute of Technology}{Rochester, NY}
                        {Bachelor of Science: Computer Engineering, 5 year
                        program\newline GPA: 3.04 --
                    \href{https://meritpages.com/sethgower}{Dean's List Spring
                '19}}{May '22}
                \resumeSubHeadingListEnd
        %---------------------------
	
		%------SKILLS---------------
            \section{Skills}
                \resumeSubHeadingListStart
                    \resumeItem{Languages}{C, VHDL, Arm Assembly,  Python, Java,
                    \LaTeX\xspace, Bash, Rust, C++, Qt5}
					\resumeItem{Tools}{Xilinx Vivado, Git, KiCad, Altera Quartus
                    II, ModelSim, GNU/Linux tools and environment}
                    \resumeItem{Hardware}{Soldering, Prototyping on
                    breadboard, Hardware design on FPGA}
                    \resumeItem{Professional Skills}{Public Speaking, Spanish (Semi Conversational)}
                \resumeSubHeadingListEnd
        %---------------------------


        %-------PROJECTS------------
            \section{Projects}
                \resumeSubHeadingListStart
                    \resumeSubsection
                        {Pipelined MIPS Processor}{VHDL}
                        {Created each stage of a MIPS processor, individually
                        tested these stages. The processor was modeled and
                        tested using VHDL and then implemented on to Basys 3 
                        FPGA. Once complete, the processor will be used to
                        calculate a portion of the Fibonacci
                        sequence}{January-May '19}
                    \resumeSubsection[github.com/SethGower/haC64]
                        {haC64}{VHDL}
                        {BrickHack V project. Myself and a partner worked to
                        emulate the Commodore 64's processor, the 6510. This
                        project is still in progress, after the processor is
                        complete, the Video Controller will be modeled.}{February
                        '19-Present}
                    \resumeSubHeadingListEnd
        %---------------------------
					
        %------WORK-EXP-------------
            \section{Work Experience}
                \resumeSubHeadingListStart
                    \resumeSubsection{D3 Engineering: Software Engineer}{Rochester, NY}
                        {Software Engineering technician at D3 Engineering, where
                        I primarily worked with Embedded Linux and C development
                        on Texas Instruments' TDA line of processors, for
                        automotive vision applications. While working with these
                        processors, I gained experience working with I2C, Linux
                    Device Tree modification, and other skills.}{January '20 - Present}
                    \resumeSubsection{iD Tech Camps: Lead Instructor}{Multiple Locations}
                        {\textit{iD Tech Camps: William and Mary, PayPal Timonium
                        and American University} \newline Taught multiple courses
                        to campers ranging from ages 7--17. Taught Java and Python
                        coding, basic Linux and circuitry on Raspberry Pi and
                        Robotics. Additionally taugh various encryption methods 
                        with Python}{June-August '18, '19} 
                    \resumeSubsection{Teaching Assistant}{Rochester, NY}
                        {\textit{RIT Computer Engineering Department}\newline
                        Teaching Assistant for the Computer Engineering
                        Department. Assisted with Intro to Computer Engineering
                        (CMPE-110), Digital Systems Design I (CMPE-160) and Applied
                        Programming in C(CMPE-380). Assisted students understand 
                        the material, graded turned in items including technical lab reports.}{August '18-Present}
                \resumeSubHeadingListEnd
        %---------------------------
		
		%-----Organizations---------
            \section{Organizations}
                \resumeSubHeadingListStart
                    \resumeSubsection[ehouse.rit.edu]
                        {Engineering House}{Rochester, NY}
                        {\textit{Positions Held: Secretary}\newline Engineering House is a social floor with an academic interest. EHouse is a place where students gather to learn and socialize through project based learning and mentor-mentee relationships.}{October '17--Present}
                    \resumeSubsection[csh.rit.edu]
                        {Computer Science House}{Rochester, NY}{Computer
                        Science House is a Special Interest House at RIT, that
                        focuses on working on projects and improving learning and
                        college life by surrounding members with like-minded
                        people. Active Member from August '17 - May '18.}
                        {August '17--Present}
				\resumeSubHeadingListEnd
        %---------------------------
    \end{document}

